\documentclass[10pt,a4paper]{article}
\providecommand{\abs}[1]{\lvert#1\rvert}
\input{AEDmacros}
\usepackage{caratula} % Version modificada para usar las macros de algo1 de ~> https://github.com/bcardiff/dc-tex
\usepackage{amsmath}
\usepackage{colonequals}
\usepackage{amsfonts}
\newcommand*{\defeq}{\mathrel{\vcenter{\baselineskip0.5ex \lineskiplimit0pt
                     \hbox{\scriptsize.}\hbox{\scriptsize.}}}%
                     =}

\titulo{Especificación y wp}
\subtitulo{En búsqueda del camino}

\fecha{\today}

\materia{Algoritmos y estructuras de datos}
\grupo{Grupo datapath}

\integrante{Andreone, Joaquin}{122/24}{jandreone06@gmail.com}
\integrante{Comerci, Lucas}{818/24}{lukicomerci@gmail.com}
\integrante{Luis, Theo}{130/23}{theoluis44@gmail.com}
\integrante{Zea, Marcos}{405/09}{zea.marcos@gmail.com}
% Pongan cuantos integrantes quieran

% Declaramos donde van a estar las figuras
% No es obligatorio, pero suele ser comodo
\graphicspath{{../static/}}

\begin{document}

\maketitle

\section{Especificaciones}

\begin{proc}{grandesCiudades}{\In ciudades : \TLista{Ciudad}}{\TLista{Ciudad}}

	\requiere{sinRepetidos(ciudades)}

    \asegura{\abs{res} \leq \abs{ciudades} \land sinRepetidos(res)}

	\asegura{\paraTodo[unalinea]{i}{\ent}{0 \leq i < \abs{ciudades}\yLuego ciudades[i]_1 \geq 50000 
    \implicaLuego ciudades[i] \in res}}
	
\end{proc}

\begin{proc}{sumaDeHabitantes}{\In menoresDeCiudades: \TLista{Ciudad}, \In mayoresDeCiudades: 
    \TLista{Ciudad}}{\TLista{Ciudad}}

    \requiere{sinRepetidos(menoresDeCiudades), sinRepetidos(mayoresDeCiudades), \\ 
    mismasCiudades(menoresDeCiudades,mayoresDeCiudades)}
    
    \asegura{\abs{res} = \abs{menoresDeCiudades}}

    \asegura{\paraTodo[unalinea]{i,j}{\ent}{(0 \leq i < \abs{menoresDeCiudades} 
    \land 0 \leq j < \abs{mayoresDeCiudades}) \yLuego \\ menoresDeCiudades[i]_0 = mayoresDeCiudades[j]_0 
    \implicaLuego \existe[unalinea]{k}{\ent}{0 \leq k < \abs{res} \land  \\ ( res[k]_0 = menoresDeCiudades[k]_0 
    \land res[k]_1 = menoresDeCiudades[i]_1 + mayoresDeCiudades[j]_1)}}}
    
\end{proc}

\begin{proc}{hayCamino}{\In distancias : \TLista{\TLista{\ent}}, \In desde: \ent, \In hasta: \ent}{\bool}

	\requiere{0 \leq desde,hasta < \abs{distancias}\land esSimetrica(distancias) \land tieneDiagonalNula(distancias)}

	\asegura{res = \True \Iff \existe[unalinea]{camino}{\TLista{\ent}}{esCamino(distancias,desde,hasta,camino)}}

\end{proc}

\begin{proc}{cantidadCaminosNSaltos}{\Inout conexion : \TLista{\TLista{\ent}}, \In n: \ent}{}
    \requiere{esMatrizDeConexion(conexion) \land n \geq 1 \land conexion = Conexion_0}

	\asegura{mismasDimensiones(conexion,Conexion_0)}

    \asegura{\existe[unalinea]{potenciasDeConexion}{\TLista{\TLista{\TLista{\ent}}}}{\abs{potenciasDeConexion} = n 
    \land \\ potenciasDeConexion[0] = Conexion_0 \land todasMismasDimensiones(potenciasDeConexion,Conexion_0) \land \\ 
    sonTodasPotenciasDe(potenciasDeConexion,Conexion_0)\land conexion = potenciasDeConexion[n-1]}}

\end{proc}

\begin{proc}{caminoMinimo}{\In origen : \ent, \In destino: \ent, \In distancias: \TLista{\TLista{\ent}}}{\TLista{\ent}}
    \requiere{0 \leq origen < \abs{distancias} \land 0 \leq destino < \abs{distancias} \land esSimetrica(distancias)}

    \asegura{esCamino(distancias,origen,destino,res) \lor res = \TLista{}} 
    \asegura{\paraTodo[unalinea]{camino}{\TLista{\ent}}{esCamino(distancias,origen,destino,camino) 
    \implicaLuego \\ longitudDeCamino(res,distancias) \leq longitudDeCamino(camino,distancias)}}

\end{proc}

\subsection{Predicados y funciones auxiliares globales}

\pred{sinRepetidos}{lista: \TLista{Ciudad}}{
    \paraTodo[unalinea]{i}{\ent}{0 \leq i < \abs{lista} \implicaLuego \paraTodo[unalinea]{j}{\ent}{
    0 \leq j < \abs{lista} \land Ciudad[i] = Ciudad[j] \implicaLuego j = i}}
}

\pred{esSimetrica}{matriz:\TLista{\TLista{\ent}}}{
    \paraTodo[unalinea]{i}{\ent}{0 \leq i < \abs{matriz} \implicaLuego \abs{matriz[i]} = \abs{matriz}} \land 
    \paraTodo[unalinea]{i,j}{\ent}{0 \leq i,j < \abs{matriz} \yLuego matriz[i][j] = matriz[j][i]}
}

\pred{tieneDiagonalNula}{matriz: \TLista{\TLista{}}}{
    \paraTodo[unalinea]{i,j}{\ent}{0 \leq i,j < \abs{matriz} \land i = j \implicaLuego matriz[i][j] = 0}
}

\pred{mismasCiudades}{\In ciudades: \TLista{Ciudad}, \In otrasCiudades: \TLista{Ciudad}}{
    \paraTodo[unalinea]{i}{\ent}{0 \leq i < \abs{ciudades} \implicaLuego \existe[unalinea]{j}{\ent}{
    0 \leq j < \abs{otrasCiudades} \yLuego ciudades[i]_0 = otrasCiudades[j]_0}}
}

\pred{esCamino}{distancias: \TLista{\TLista{\ent}}, desde: \ent, hasta: \ent, camino: \TLista{\ent}}{
    desde \in camino \land hasta \in camino \land desde \neq hasta \land \paraTodo[unalinea]{i}{\ent}{
    0 \leq i < \abs{camino} \implicaLuego 0 \leq camino[i] < \abs{distancias}} \land \\
    \paraTodo[unalinea]{i}{\ent}{0 \leq i < \abs{camino}-1 \implicaLuego 
    distancias[camino[i]][camino[i+1]] \neq 0}
}

\pred{esMatrizDeConexion}{matriz: \TLista{\TLista{\ent}}}{
    esSimetrica(matriz) \land tieneDiagonalNula(matriz) \land 
    \paraTodo[unalinea]{i,j}{\ent}{0 \leq i,j < \abs{matriz} \land i \neq j \implicaLuego \\
    (matriz[i][j] = 1 \lor matriz[i][j] = 0)}
}

\pred{esProductoDe}{matrizA: \TLista{\TLista{\ent}}, matrizB: \TLista{\TLista{\ent}}, matriz: \TLista{\TLista{\ent}}}{
    mismasDimensiones(matrizA,matrizB) \land mismasDimensiones(matrizA,matriz) \land \\ 
    \paraTodo[unalinea]{i,j}{\ent}{0 \leq i,j < \abs{matriz} \implicaLuego 
    matriz[i][j] = \sum_{k=0}^{\abs{A}-1}matrizA[i][k] \cdot matrizB[k][j]}
}

\pred{mismasDimensiones}{matrizA: \TLista{\TLista{\ent}}, matrizB: \TLista{\TLista{\ent}}}{
    \abs{matrizA} = \abs{matrizB} \land \paraTodo[unalinea]{i}{\ent}{0 \leq i < \abs{matrizA} 
    \implicaLuego \abs{matrizA[i]} = \abs{matrizB[i]}}
}

\pred{todasMismasDimensiones}{matrices: \TLista{\TLista{\TLista{\ent}}}, matriz: \TLista{\TLista{\ent}}}{
    \paraTodo[unalinea]{i}{\ent}{0 \leq i < \abs{matrices} \implicaLuego mismasDimensiones(matrices[i],matriz)}
}

\pred{sonTodasPotenciasDe}{matrices: \TLista{\TLista{\TLista{\ent}}}, matriz: \TLista{\TLista{\ent}}}{
    \paraTodo[unalinea]{i}{\ent}{1 \leq i < \abs{matrices} \implica esProductoDe(matriz,matrices[i-1],matrices[i])}
}

\aux{longitudDeCamino}{camino: \TLista{\ent}, distancias: \TLista{\TLista{\ent}}}{\ent}{ \\
\displaystyle \sum_{i = 0}^{\abs{camino}-2} distancias[camino[i]][camino[i+1]]
}

\section{Demostraciones de correctitud}

\begin{lstlisting}[mathescape=true][caption={}label=code:for]
S1: res $\colonequals$ 0;
S2: i $\colonequals$ 0;
S3: while (i < ciudades.length) do
S4:     res $\colonequals$ res + ciudades[i].habitantes 
S5:     i = i + 1
    endwhile
\end{lstlisting}

Como se ve en la implementacion de la especificacion, el programa consta solo de un ciclo while, por lo que para 
demostrar su correctitud debemos usar el ``Teorema del Invariante'' y el ``Teorema de terminación de un ciclo". Se 
tienen que cumplir estas condiciones:

\begin{enumerate}
    \item $P_c \Longrightarrow I $
    \item $\{I \wedge B \} S \{I\} $
    \item $I \wedge \neg B \Longrightarrow Q_c $
    \item $\{I \wedge B \wedge v_0 = fv \} S \{ fv < v_0 \} $
    \item $I \wedge fv \leq 0 \Longrightarrow \neg B $
\end{enumerate}

Basado en la especificacion y la implementacion propongo:
\begin{itemize}
    \item $P_c \equiv res = 0 \wedge i = 0 $
    \item $B \equiv i < |ciudades| $
    \item $S_1 \equiv res := res + ciudades[i]_1 $
    \item $S_2 \equiv i := i + 1 $
    \item $Q_c \equiv res = \sum_{j=0}^{|ciudades|-1}ciudades[j]_1 \wedge res > 50000 $ 
    \item $I \equiv 0 \leq i \leq |ciudades| \wedge res = \sum_{j=0}^{i-1}ciudades[j]_1 $
    \item $ fv = |ciudades| - i$
\end{itemize}

\subsection{$P_c \Longrightarrow I $}
Reemplazo:
$$ res = 0 \wedge i = 0 \Longrightarrow 0 \leq i \leq |ciudades| \wedge res = \sum_{j=0}^{i-1}ciudades[j]_1 $$
$$ 0 \leq i \leq |ciudades| \equiv 0 \leq 0 \leq |ciudades| \equiv True$$ 
$$ res = \sum_{j=0}^{i-1}ciudades[j]_1 \equiv 0 = \sum_{j=0}^{0-1}ciudades[j]_1 = 0 \equiv True $$
Por lo tanto, se cumple $P_c \Longrightarrow I $

\subsection{$\{I \wedge B \} S \{I\} $}
Para que la tripla de Hoare sea valida, la precondición debe implicar la ``Weakest precondition" del código y la 
postcondición. Es decir:
$$ \{I \wedge B \} S \{I\} \Longleftrightarrow \{I \wedge B \} \Longrightarrow_L wp(S, I) $$
\begin{align*}
    wp(S, I) &\equiv wp(S_1, wp(S_2, I)) \equiv wp(S_1, wp(i := i + 1, 0 \leq i \leq |ciudades| \wedge 
    res = \sum_{j=0}^{i-1}ciudades[j]_1)) \\
    &\equiv wp(S_1, def(i+1) \wedge_L 0 \leq i+1 \leq |ciudades| \wedge res = \sum_{j=0}^{i+1-1}ciudades[j]_1) \\
    &\equiv wp(res := res + ciudades[i]_1, 0 \leq i+1 \leq |ciudades| \wedge res = \sum_{j=0}^{i}ciudades[j]_1) \\ 
    &\equiv def(res) \wedge_L def(ciudades[i]_1) \wedge 0 \leq i+1 \leq |ciudades| \wedge 
    res + ciudades[i]_1 = \sum_{j=0}^{i}ciudades[j]_1 \\
    &\equiv 0 \leq i < |ciudades| \wedge_L 0 \leq i+1 \leq |ciudades| \wedge 
    res + ciudades[i]_1 = \sum_{j=0}^{i}ciudades[j]_1
\end{align*}

Combino los intervalos de $i$ y resto en ambas partes de la igualdad $ciudades[i]_1$ para transformar el rango de la 
sumatoria

$$ \equiv 0 \leq i < |ciudades| \wedge_L res = \sum_{j=0}^{i-1}ciudades[j]_1 $$

Como $I \wedge B \equiv wp(S,I)$, por lo tanto $\{I \wedge B \} \Longrightarrow wp(S,I) $ que era lo que 
necesitaba para probar que la tripla de Hoare sea válida.

\subsection{$I \wedge \neg B \Longrightarrow Q_c $}
Reemplazo:
$$ I \wedge \neg B \equiv 0 \leq i \leq |ciudades| \wedge res = \sum_{j=0}^{i-1}ciudades[j]_1 \wedge i \geq |ciudades|$$
Si $i \geq |ciudades| \wedge i \leq |ciudades|$ entonces $i = |ciudades|$
$$ i = |ciudades| \wedge res = \sum_{j=0}^{i-1}ciudades[j]_1 \Longrightarrow 
res = \sum_{j=0}^{|ciudades|-1}ciudades[j]_1 \wedge res > 50000 \equiv Q_c $$
Esto se cumple, ya que por especificación al menos una de ellas es grande, es decir, supera los 50.000 habitantes.

\subsection{$\{I \wedge B \wedge v_0 = fv \} S \{ fv < v_0 \}$}
Para que la tripla de Hoare sea válida, la precondición debe implicar la ``Weakest precondition" del programa y la 
postcondicion. Es decir:
$$\{I \wedge B \wedge v_0 = fv \} S \{ fv < v_0 \} \Longleftrightarrow \{I \wedge B \wedge v_0 = fv \} 
\Longrightarrow wp(S, fv < v_0)$$

Calculo wp:
\begin{align*}
    wp(S, fv < v_0) &\equiv wp(S_1, wp(S_2, fv < v_0)) \equiv wp(S_1, wp(i := i + 1, |ciudades| - i < v_0)) \\
    &\equiv wp(res := res + ciudades[i]_1, def(i+1) \wedge_L |ciudades| - (i+1) < v_0)  \\
    &\equiv def(ciudades[i]) \wedge_L |ciudades| - i - 1 < v_0 \\
    &\equiv 0 \leq i < |ciudades| \wedge_L |ciudades| - i - 1 < v_0
\end{align*}
Y desarrollo la precondición:

$$\{I \wedge B \wedge v_0 = fv \} \equiv 0 \leq i \leq |ciudades| \wedge res = \sum_{j=0}^{i-1}ciudades[j]_1 \wedge 
i < |ciudades| \wedge |ciudades| - i = v_0 $$
Ignoro lo que corresponde a $res$ porque no lo necesito para la demostración, y combino los rangos de i:
$$ 0 \leq i < |ciudades| \wedge |ciudades| - i = v_0 $$
Entonces:
$$\{I \wedge B \wedge v_0 = fv \} \Longrightarrow wp(S, fv < v_0) \Longleftrightarrow $$
$$ 0 \leq i < |ciudades| \wedge |ciudades| - i = v_0 \Longrightarrow 0 \leq i < |ciudades| \wedge_L 
|ciudades| - i - 1 < v_0 \Longleftrightarrow$$
$$ |ciudades| - i - 1 < |ciudades| - i \Longleftrightarrow -1 < 0 \equiv True $$

Por lo tanto, la tripla de Hoare es válida.

\subsection{$I \wedge fv \leq 0 \Longrightarrow \neg B $}

Desarrollo:
$$ I \wedge fv \leq 0 \equiv 0 \leq i \leq |ciudades| \wedge res = \sum_{j=0}^{i-1}ciudades[j]_1 \wedge 
|ciudades| - i \leq 0 \Longrightarrow \neg B$$
Ignoro lo que corresponde a $res$ porque no lo necesito para la demostración:

$$0 \leq i \leq |ciudades| \wedge |ciudades| - i \leq 0 \Longrightarrow i \geq |ciudades| \equiv$$
$$0 \leq i \leq |ciudades| \wedge |ciudades| \leq i \Longrightarrow i \geq |ciudades| \equiv$$
$$ i = |ciudades| \Longrightarrow i \geq |ciudades| \equiv True$$

\end{document}